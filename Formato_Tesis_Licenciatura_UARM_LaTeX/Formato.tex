\documentclass[12pt, a4paper]{article}
\usepackage[utf8]{inputenc} %Codificación tipográficas
\usepackage[spanish]{babel} %Idioma
\usepackage{amsmath} %Insertar ecuaciones
\usepackage{amsfonts}
\usepackage{amssymb}
\usepackage{graphicx} %Insertar figuras
\usepackage[svgnames]{xcolor}
\usepackage{pstricks}
\usepackage{marginnote}
\usepackage{float}
\usepackage{multicol} %Muliples columnas
\usepackage{xspace}
\usepackage{makeidx}
\usepackage[natbibapa]{apacite} %Referencias estilo APA
\usepackage{wrapfig} %Figura a lado del texto
\usepackage{makecell} % Para tablas
\usepackage{multirow} % Para tablas
\usepackage{booktabs} % Para linea de tablas

%AJUSTES DE PÁGINA
\usepackage[papersize={21cm, 29.7cm}, centering, left=3.5cm, right=2.5cm, top=2.5cm, bottom=2.5cm]{geometry} 

%ENCABEZADOS Y PIE DE PÁGINA
\usepackage{fancyhdr}
\pagestyle{fancy}
\fancyhead{} % contenido de encabezado. por omisión: sección
\lhead{} % superior izquierda
\chead{} % superior centro
\rhead{} % superior derecha
\lfoot{} % inferior izquierda
\cfoot{} % inferior centro
\rfoot{\thepage} % inferior derecha
\renewcommand{\footrulewidth}{0pt}% línea arribe del encabezado
\renewcommand{\headrulewidth}{0pt}% línea debajo del encabezado

%TIPOS DE LETRA
%\renewcommand*{\familydefault}{\sfdefault} %Arial
\usepackage[T1]{fontenc}
\usepackage{times} %Times New Roman

%HIPERVÍNCULOS
\usepackage[hidelinks=true
colorlinks=black,
linkcolor=black,
urlcolor=black,
citecolor=blue,
anchorcolor=blue]{hyperref}
%\usepackage[table, xcdraw]{xcolor}
\usepackage[normalem]{ulem}
\hypersetup{colorlinks, urlcolor=blue}
\makeatletter
\DeclareUrlCommand\ULurl@@{%
	\def\UrlFont{\rmfamily\color{blue}}%
	\def\UrlLeft{\uline\bgroup}%
	\def\UrlRight{\egroup}}
\def\ULurl@#1{\hyper@linkurl{\ULurl@@{#1}}{#1}}
\DeclareRobustCommand*\ULurl{\hyper@normalise\ULurl@}
\makeatother

%AJUSTES ADICIONALES
\renewcommand{\baselinestretch}{1.5}%Interlineado
\setlength{\parskip}{0pt}%Espacio entre párrafos
%\setlength{\parindent}{1em} %Sangría  primera linea de párrafos
%\setcounter{tocdepth}{2} %Tabla de contenidos (2 = título y subtítulo)




	



\begin{document}

\newpage
\break

\thispagestyle{empty} %oculta número de página
	\begin{center}
		
	\textbf{\large UNIVERSIDAD ANTONIO RUIZ DE MONTOYA}
	
	\vspace{1cm}
	Faculdad de Ciencia Sociales
	
	\vspace{1cm}
	
	\includegraphics[width=0.45\textwidth]{Logos_Ruiz/logo_ruiz.png}
	
	\vspace{1cm}
	
	\textbf{\large TÍTULO DEL TRABAJO}
	
	\vspace{1cm}
	
	Trabajo de Suficiencia Profesional para optar al Título Profesional de Licenciado en ...
	
	\vspace{1.cm}
	
	\hspace{-11cm}Presenta el Bachiller\\
	\vspace{1cm}
	\textbf{\large NOMBRES APELLIDO 1 APELLIDO 2}
	
	\vspace{1cm}
	
	\textbf{Presidente: Nombre Apellido 1 Apellido2\\
		Asesor: Nombre Apellido 1 Apellido2\\
		Lector 1: Nombre Apellido 1 Apellido2\\
		Lector 2: Nombre Apellido 1 Apellido2}
	
	\vspace{2.5cm}
	
	\textbf{Lima - Perú \\ Mes 2021}
\end{center}
\thispagestyle{empty} %oculta número de página
\begin{flushright}
	
	\medspace
	\vspace{1.5cm}
	
	\textbf{\large EPÍGRAFE}\\
	
	\vspace{1cm}
	
	Desarrollo del contenido del epígrafe (máximo 100 palabras)
	
\end{flushright}
\thispagestyle{empty} %oculta número de página
\begin{flushright}
	
	\medspace
	\vspace{1.5cm}
	
	\textbf{\large DEDICATORIA}\\
	
	\vspace{1cm}
	
	Desarrollo del contenido de la dedicatoria (máximo 100 palabras)
	
\end{flushright}
\thispagestyle{empty} %oculta número de página
\begin{flushright}
	
	\medspace
	\vspace{1.5cm}
	
	\textbf{\large AGRADECIMIENTO}\\
	
	\vspace{1cm}
	
	Desarrollo del contenido del Agradecimiento (máximo 100 palabras)
	
\end{flushright}
\thispagestyle{empty} %oculta número de página
\medspace
\vspace{1.5cm}
\begin{center}
	\large \bf RESUMEN
\end{center}

\vspace{1cm}

Desarrollo del contenido 


\vspace{1cm}

\textbf{Palabras clave:} palabra1, palabra2, etc.


\break

\thispagestyle{empty} %oculta número de página
\medspace
\vspace{2cm}

\begin{center}
	\large \bf ABSTRACT
\end{center}

\vspace{1cm}

Desarrollo del contenido en inglés


\vspace{1cm}

\textbf{Keywords:} word1, word2, etc.







\break

\medspace
\vspace{1.5cm}
\begin{center} %Tabla de contenidos
	\thispagestyle{empty} %oculta número de página
	\renewcommand*\contentsname{} %elimina el título
	\textbf{\large TABLA DE CONTENIDOS}
	\vspace{-1cm}
	\tableofcontents
\end{center}


\break

\medspace
\vspace{1.5cm}
\begin{center}
	\thispagestyle{empty}
	\renewcommand{\listtablename}{}
	\textbf{\large ÍNDICE DE TABLAS}
	\vspace{-1cm}
	\listoftables
\end{center}




\break

\medspace
\vspace{1.5cm}
\begin{center}
	\thispagestyle{empty}
	\renewcommand{\listfigurename}{}
	\textbf{\large ÍNDICE DE FIGURAS}
	\vspace{-1cm}
	\listoffigures
\end{center}


\medspace
\vspace{1.5cm}
\begin{center}
	\addcontentsline{toc}{section}{INTRODUCCIÓN}
	\section*{\large INTRODUCCIÓN}
\end{center}

\vspace{1cm}

Desarrollo de contenido
\medspace
\vspace{1.5cm}
\begin{center}
	\addcontentsline{toc}{section}{CAPÍTULO I: TÍTULO DEL CAPÍTULO}
	\section*{\large CAPÍTULO I: TÍTULO DEL CAPÍTULO}
\end{center}

\vspace{1cm}

Desarrollo de contenido

\addcontentsline{toc}{subsection}{1.1. Subsección}
\subsection*{\normalsize 1.1. Subsección}


\addcontentsline{toc}{subsubsection}{1.1.1. Subsección}
\subsection*{\normalsize 1.1.1. Sububsección}
\medspace
\vspace{1.5cm}
\begin{center}
	\addcontentsline{toc}{section}{CAPÍTULO II: TÍTULO DEL CAPÍTULO}
	\section*{\large CAPÍTULO II: TÍTULO DEL CAPÍTULO}
\end{center}

\vspace{1cm}

Desarrollo de contenido
\addcontentsline{toc}{subsection}{2.1.  Sección}
\subsection*{\normalsize 2.1.  Subección}

Ejemplo para insertar figura:

\begin{figure}[h!]
	\centering
	\includegraphics[width=0.45\textwidth]{Logos_Ruiz/logoruiz_insignia}
	\caption[Descripción que aparece en el índice de figuras]{Descripción que aparece debajo la figura}
	\label{fig:insignia_ruiz}
\end{figure}

	
\medspace
\vspace{1.5cm}
\begin{center}
	\addcontentsline{toc}{section}{CAPÍTULO III: TÍTULO DEL CAPÍTULO}
	\section*{\large CAPÍTULO III: TÍTULO DEL CAPÍTULO}
\end{center}

\vspace{1cm}

Desarrollo de contenido
\addcontentsline{toc}{subsection}{3.1. Subsección}
\subsection*{\normalsize 3.1. Subsección}

Ejemplo insertar tablas:

\begin{table}[h!]
	\centering
	\caption[Descripción que aparece en el índice de tablas]{Descripción que aparece arriba de la tabla}
	\begin{tabular}{c c c}
		\toprule
		item 1 & item 2 & item 3\\
		\toprule
		item 4 & item 5 & item 6\\
		item 7 & item 8 & item 9\\
		item 10 & item 11 & item 12\\
		\toprule
		\multicolumn{3}{c}{Esta tabla es un ejemplo que debe eliminarse}\\
		\multicolumn{3}{c}{Esta tabla es un ejemplo que debe eliminarse}	
	\end{tabular}
\end{table}

\medspace
\vspace{1.5cm}
\begin{center}
	\addcontentsline{toc}{section}{CAPÍTULO IV: TÍTULO DEL CAPÍTULO}
	\section*{\large CAPÍTULO IV: TÍTULO DEL CAPÍTULO}
\end{center}

\vspace{1cm}

Desarrollo de contenido
\addcontentsline{toc}{subsection}{4.1. Subsección}
\subsection*{\normalsize 3.1. Subsección}

A continuación se presenta como ejemplo la expresión matemática del cálculo de la media muestral en la ecuación \ref{media}.

\begin{equation}
\label{media}
	\LARGE
	\bar{X} = \sum_{i = 1}^{n} \frac{x_i}{n}
\end{equation}


\medspace
\vspace{1.5cm}
\begin{center}
	\addcontentsline{toc}{section}{Conclusiones}
	\section*{\large CONCLUSIONES}
\end{center}

\vspace{1cm}

Desarrollo de contenido	
\medspace
\vspace{1.5cm}
\begin{center}
	\addcontentsline{toc}{section}{Recomendaciones}
	\section*{\large RECOMENDACIONES}
\end{center}

\vspace{1cm}

Desarrollo de contenido


\medspace
\vspace{2cm}
\begin{center}
	\addcontentsline{toc}{section}{Referencias bibliográficas}
	\section*{\large REFERENCIAS BIBLIOGRÁFICAS}
\end{center}
\vspace{-1cm}
\renewcommand{\refname}{} %modifica título por defecto
\bibliographystyle{apacite}
\bibliography{referencias}	

\break

\medspace
\vspace{9.5cm}
\begin{center}
	\LARGE \bf ANEXOS
\end{center}

\medspace
\vspace{1.5cm}
\begin{center}
	\addcontentsline{toc}{section}{Anexos}
	\section*{ANEXO N$^\circ$ 1: TÍTULO DEL ANEXO}
\end{center}

\vspace{1cm}

Desarrollo de contenido

\end{document}



